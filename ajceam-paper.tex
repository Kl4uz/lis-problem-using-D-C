%% This is file AJCEAM-paper.tex is the template file for publications
%% in the "Academic Journal on Computing, Engineering and Applied Mathematics
%% from Universidade Federal do Tocatins, Brazil,
%% https://sistemas.uft.edu.br/periodicos/index.php/AJCEAM/index
%% This file was originally written by Tiago Almeida.
%% First revision: 2019-10-29
%% Second revision:
%%
%% History:
%%           2019-10-29 - First version by Tiago Almeida <tiagoalmeida@uft.edu.br>.
%%           2020-04-09 - English revision by Caio Machado <caio.machado@uft.edu.br>
%%

% Options
%     eng:   English Language
%     por:   Portuguese Language
%     blind: The version for reviewers is compiled (author data is hidden)


\documentclass[eng]{ajceam-class}
%
% Publication Title
\title{Author instructions for the Academic Journal on Computing, Engineering and Applied Mathematics}

% Publication Title (Portuguese only)
\titulo{Instruções para os autores da Academic Journal on Computing, Engineering and Applied Mathematics}

% Short title for the header (copy the main title if it is not too long)
\shorttitle{Author instructions for the AJCEAM}
       
% Authors
\author[1]{Lucas C. Farias}
\author[2]{Vicente B. Gregório}

% Author Affiliations
\affil[1]{University Name, Department Name or Institute, State, Country}
\affil[2]{(Other) University Name, Department Name or Institute, State, Country}
\affil[3]{(Other) University Name, Department Name or Institute, State, Country}

% Surname of the first author of the manuscript
\firstauthor{SurnameA, SurnameB and SurnameC}

%Contact Author Information
\contactauthor{Name A. Surname}           % Name and surname of the contact author
\email{name.surname@email.com}            % Contact Author Email

% Publication data (will be defined in the edition)
\thisvolume{XX}
\thisnumber{XX}
\thismonth{Month}
\thisyear{20XX}
\receptiondate{dd/mm/aaaa}
\acceptancedate{dd/mm/aaaa}
\publicationdate{dd/mm/aaaa}

% Place your particular definitions here
\newcommand{\vect}[1]{\mathbf{#1}}  % vectors

% Insert here the abstract in English language
\abstract{This article proposes an improvised divide and conquer algorithm that presents an optimal solution for all cases of the Longest Increasing Subsequence (LIS) problem. This problem is related to the Longest Common Subsequence (LCS) problem, as any solution that solves an LIS also solves an LCS when applied simultaneously to two or more arrays. Algorithms used for this type of problem can have various applications such as pattern identification and data compression.}

\abstractpt{Neste artigo é proposto um algoritmo de divisão e conquista que apresenta uma solução ótima para todos os casos para o problema da maior subsequência crescente (LIS). Este problema pode ser relacionado com o problema da Maior Subsequência Comum (LCS), pois qualquer solução que resolva uma LIS resolve também um LCS, basta aplicar simultaneamente em duas ou mais arrays. Os algoritmos usados para este tipo de problema podem ter várias aplicações como a identificação de padrões e compressão de dados.}

% Insert here the keywords of your work in English language
\keywords{Divide and Conquer, Longest Increasing Subsequence (LIS), Algorithm, Data Compression, Pattern Identification} % Palavras-chave sugeridas

% Insert here the abstract in Portuguese language
% Start document
\begin{document}

\keywords{Divide and Conquer, Longest Increasing Subsequence (LIS), Algorithm, Data Compression, Pattern Identification} % Palavras-chave sugeridas

\resumo{

% Include title, authors, abstract, etc.
\maketitle
\thispagestyle{fancy}
\printcontactdata

% Main body of manuscript
\section{Introduction}
\firstword{O}{}% Capital letter in first word
presente trabalho se insere no contexto da análise de algoritmos e estruturas de dados, com foco na técnica de Divisão e Conquista. Conforme o objetivo desta atividade, o grupo selecionou o artigo científico ``\titleen{Improvised Divide and Conquer Approach for the LIS Problem}'' de Rani e Rajpoot (2018) \cite{Rani2018}, que propõe uma solução otimizada para o problema da Maior Subsequência Crescente (LIS - Longest Increasing Subsequence).

O problema LIS consiste em encontrar a subsequência de maior comprimento em uma dada sequência de números, de forma que os elementos da subsequência estejam em ordem estritamente crescente.

A escolha deste artigo se justifica pela clareza na aplicação do paradigma de Divisão e Conquista, que consiste na divisão do problema em $n$ subproblemas menores, que por sua vez são resolvidos individualmente, e posteriormente têm suas soluções combinadas para resolver o problema original. Para uma sequência $X$ de tamanho $n$, o artigo descreve a divisão em duas subsequências $x^1$ e $x^2$ de $X$.

O restante deste trabalho está estruturado da seguinte forma: a Seção \ref{sec:revisao} apresenta uma breve revisão bibliográfica sobre o problema LIS e a técnica de Divisão e Conquista; a Seção \ref{sec:metodologia} detalha a metodologia proposta no artigo de Rani e Rajpoot (2018); a Seção \ref{sec:implementacao} descreve a implementação do algoritmo; e a Seção \ref{sec:conclusao} apresenta as conclusões e trabalhos futuros.\verb|\palavraschave{}|, respectively.

\section{Bibliographic citations}
\label{sec:revisao}
O problema LIS é um problema clássico em ciência da computação, com aplicações em áreas como bioinformática (alinhamento de sequências), compressão de dados e identificação de padrões.

\subsection{Problema da Maior Subsequência Crescente (LIS)}
[... Detalhar a definição formal do problema LIS e as abordagens tradicionais (e.g., Programação Dinâmica) ...]

\subsection{Técnica de Divisão e Conquista}
[... Detalhar o paradigma Divisão e Conquista (dividir, conquistar, combinar) e exemplos clássicos (e.g., Merge Sort) ...]


\section{Metodologia Proposta (Rani e Rajpoot, 2018)}
\label{sec:metodologia}
[... Descrever em detalhes o algoritmo proposto no artigo, focando nas etapas de divisão, subproblemas e combinação. Usar exemplos práticos, conforme solicitado nas instruções. ...]


\section{Implementação do Algoritmo}
\label{sec:implementacao}
Nesta seção, é apresentada a implementação do algoritmo de Divisão e Conquista proposto.

\subsection{Linguagem de Programação e Ambiente}
[... Informar a linguagem e o ambiente de desenvolvimento utilizados. ...]

\subsection{Código Fonte}
[... Apresentar o código fonte do algoritmo e, se aplicável, da versão otimizada. ...]

\section{Resultados e Discussão}
[... Apresentar os resultados da execução, comparações com outras abordagens (se o artigo original as tiver) e discutir as otimizações. ...]

\section{Conclusão}
\label{sec:conclusao}
[... Concluir o trabalho, resumindo os achados e sugerindo trabalhos futuros. ...]


\section*{Agradecimentos}
[... Opcional. ...]

% Include references
\insertbibliography{References}

\end{document}
